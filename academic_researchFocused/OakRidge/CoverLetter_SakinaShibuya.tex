
\documentclass[12pt]{letter}
\usepackage[utf8]{inputenc}
\usepackage[left=0.75in,right=0.75in,bottom=0.75in,top=0.25in]{geometry} 
\usepackage{graphicx} \graphicspath{{../../}}
\usepackage{multirow}
\usepackage{tabularx}

\begin{document}

\begin{tabularx}{\textwidth}{Xr}
\multirow{4}{*}{\includegraphics[height=3\baselineskip]{logo_cropped.pdf}} &  \\
& Taylor Hall \\
& 427 Lorch Street \\
& Madison, WI 53706 \\
[-1.8ex]\\
\\
\end{tabularx}

Environmental Sciences Division \hfill \today \\
Oak Ridge National Laboratory

\medskip

Dear Hiring Committee Members,

I am writing to apply for the position of the Postdoctoral Research Associate
at the Environmental Sciences Division of Oak Ridge National Laboratory (ORNL). 
I am an applied microeconomist who integrates rigorous empirical methods to address pressing issues related to gender, conflict, and economic development.
I will complete my Ph.D. in Agricultural and Applied Economics at the University of Wisconsin-Madison in May 2025. 
I am particularly drawn to ORNL for the possibility of applying my expertise in applied econometrics to policy-relevant research on sustainability and energy. 
While my graduate school is in Madison, Wisconsin, I currently live in Gatlinburg, Tennessee.
%I am particularly drawn to [School Name] due to its [Insert].

My job market paper, “Effects of Military Bases on Women in Colombia,” co-authored with Felipe Parra, investigates the causal impacts of military base presence 
on sexual violence, fertility, and child support disputes in rural Colombia. Using a novel dataset constructed from various sources and employing an event-study approach, 
we leverage the temporal and geographical variation in military base placements driven by Colombia's military expansion from 2000 to 2016. This allows us to identify 
the causal effects of military interventions, revealing a 72\% increase in registered cases of sexual violence over 15 years. 
This research contributes to understanding the unintended consequences of state security interventions and provides critical insights 
for policymakers working in conflict-affected settings.

In addition to my focus on conflict and gender dynamics, my broader research agenda explores the economic barriers women face in labor markets. 
In a project in Pakistan's garment sector, I analyze how social norms affect employers' decisions 
to hire women. This research, supported by funding from the International Growth Centre (IGC) and Private Enterprise Development in Low-Income Countries (PEDL), 
aims to develop policy interventions to increase female labor force participation, particularly in environments constrained by traditional gender norms.

In the coming years, I am committed to securing funding from a diverse range of public, private, and non-profit sources, and 
will use internal and external funding to support the research agenda of the Environmental Sciences Division.

Thank you for considering my application. 
I can be reached by telephone at +1 (917) 969-5420, and by email at sshibuya2@wisc.edu.

\bigskip

\includegraphics[height=4\baselineskip]{signature.png}  \\

\vspace*{-6.5\baselineskip}Sincerely, 

\vspace{2.5\baselineskip}Sakina Shibuya

\end{document}
